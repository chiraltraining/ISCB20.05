
\begin{frame}[t]{A Quick Review of Data and Variables--1}
	\begin{itemize}
		\item \textbf{Variable}
		\begin{itemize}
			\item[--]  A characteristic taking on different values. 
		\end{itemize}
	\end{itemize}
	
	\begin{itemize}
		\item \textbf{Random Variable}
		\begin{itemize}
			\item[--] A variable taking on different possible values as a 
			result of chance factors. 
		\end{itemize}
	\end{itemize}

		\begin{itemize}
		\item \textbf{Quantitative or Numerical Data}
		\begin{itemize}
			\item[--] Implies amount or quantity
		\end{itemize}
	\end{itemize}
	
	\begin{itemize}
		\item \textbf{Discrete}
		\begin{itemize}
			\item[--] Random variable with values that comprise a countable
			set
			\item [--]  There can be gaps in its possible values
		\end{itemize}
	\end{itemize}	
\end{frame}
% Slide-2
\begin{frame}[t]{A Quick Review of Data and Variables--2}
	\begin{itemize}
		\item \textbf{Continuous}
		\begin{itemize}
			\item[--] Random variable with values comprising an interval of
			real numbers
			\item [--]  There are no gaps in its possible values
		\end{itemize}
	\end{itemize}
	
	
	\begin{itemize}
		\item\textbf{Qualitative or Categorical Data}
		\begin{itemize}
			\item[--] Implies attribute or quality
		\end{itemize}
	\end{itemize}

	\begin{itemize}
		\item \textbf{Nominal}
		\begin{itemize}
			\item[--] Classifications based on names
		\end{itemize}
	\end{itemize}
		
	\begin{itemize}
		\item \textbf{Ordinal}
		\begin{itemize}
			\item[--] Classifications based on an ordering or ranking
		\end{itemize}
	\end{itemize}
\end{frame}
\begin{frame}[t]{Descriptive Statistics}
	\begin{itemize}
		\item Also known as Exploratory data analysis(EDA) 
		\item Summarize data as it is 
		\item Do not posit any hypothesis about data
		\item Do not try to fit models to data
		\item Very important initial step
		\item Often neglected 
		\item Detect outliers
		\item Plan how to prepare data
		\item Precursor to feature engineering
		\item Descriptive visualization
	\end{itemize}
\end{frame}

% Slide-n
\begin{frame}[t]{Scale of Measurement--1}
	\textbf{Counts}
	\begin{itemize}
		\item Numbers represented by whole numbers. 
		\begin{itemize}
			\item [--]For example, number of births, number of relapses
		\end{itemize}
	\end{itemize}
	
	\textbf{Interval}
	\begin{itemize}
		\item The same distances or intervals between values are equal.
		\begin{itemize}
			\item [--]For example, temperature, altitude
		\end{itemize}
	\end{itemize}

	\textbf{Ratio}
	\begin{itemize}
		\item The same ratios of values are equal.
		\begin{itemize}
			\item [--]For example, weight, height, time, hospital length of
			stay
			\item [--]A true zero point indicates the absence of the quantity
			being measured
		\end{itemize}
	\end{itemize}

\end{frame}

% Slide-n
\begin{frame}[t]{Scale of Measurement--2}
	\textbf{Nominal}
	\begin{itemize}
		\item Classifications based on names.
		\begin{itemize}
			\item Binary or dichotomous
			\begin{itemize}
				\item [--]For example, gender, alive or dead
			\end{itemize}
		\end{itemize}
		
		\begin{itemize}
			\item Polychotomous or polytomous
			\begin{itemize}
				\item[--] For example, marital status, ethnicity
			\end{itemize}
		\end{itemize}
	\end{itemize}

\textbf{Ordinal}
\begin{itemize}
	\item Classifications based on an ordering or ranking
	\begin{itemize}
		\item [--]For example, ratings, preferences
	\end{itemize}
\end{itemize}
\end{frame}

% Slide-n
\begin{frame}[t]{Methods for Organizing and Summarizing Data}
	\begin{itemize}
		\item \textbf{Numerical Summary}
		\begin{itemize}
			\item Frequency Distributions
			\item Measure of Central Tendency
			\item Measure of Spread or Dispersion
			\item Correlation and Covariance
			\item Confidence Intervals
			\item Skewness and Kurtosis 
		\end{itemize}
	\end{itemize}

	\begin{itemize}
		\item \textbf{Graphical Summary}
	\begin{itemize}
		\item Tables
		\item Histograms
		\item Bar Charts
		\item Box-and-whiskers plots
		\item Scatter Plots
		\item Pie Chart 
	\end{itemize}
	\end{itemize}
\end{frame}
