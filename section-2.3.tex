% Slide-3.2
\begin{frame}[t]{Measures of Central Tendency}
		\begin{itemize}
		\item Average (Mean)
		\item Median
		\item Mode
		\item Other infrequently used measures
		\begin{itemize}
			\item Geometric Mean
			\item Harmonic Mean
		\end{itemize}
	\end{itemize}
\end{frame}
% Slide-3.2.1
\begin{frame}[t]{Mean}
	\begin{itemize}
		\item Single best value to represent data
		\item Need not actually be data point itself
		\item Considers every point in data
		\item Discrete as well as continuous data
		\item Vulnerable to outliers
	\end{itemize}
\end{frame}


\begin{frame}[t]{Arithmetic Mean of a Dataset}
	\begin{itemize}
		\item The arithmetic mean is calculated as the sum of the values 
		divided by the total number of values, referred to as $n$.\\ 
		$$AM =  \frac{(x_1 + x_2 + … + x_n)}{n} $$
		
		\item A more convenient way to calculate the arithmetic mean is to 
		calculate the sum of the values and to multiply it by the reciprocal of 
		the number of values ($\frac{1}{n}$) \\ 
		$$ AM = (\frac{1}{n}) \times (x_1 + x_2 + … + x_n) $$
		\item The arithmetic mean is appropriate when all values in the data 
		sample have the same units of measure, e.g. all numbers are heights, or 
		dollars, or miles, etc.
		\item When calculating the arithmetic mean, the values can be positive, 
		negative, or zero.
	\end{itemize}
\end{frame}

% Slide-3.2.2
\begin{frame}[t]{Arithmetic Mean of a Dataset--1}
	\textbf{Example:} Five systolic blood pressures (mmHg) (n = 5) \\ 
	120, 80, 90, 110, 95
	
	$$
	Mean = 
	 \frac{120+80+90+110+95}{5}
	= \frac{495}{5}
	= 99mmHg 
	$$
	
	$$
	Mean = \overline{x}  = \frac{\sum x_i }{n}
	$$
	
	\begin{itemize}
		\item $	\overline{x} = $ mean of a dataset
		\item $	x_i =$ data points 
		\item $	n =$ number of sample 
	\end{itemize}
\end{frame}

\begin{frame}[t]{Arithmetic Mean of a Dataset--2}
	\textbf{Example:} Five systolic blood pressures (mmHg) (n = 5) \\ 
	120, 80, 90, 110, 95
	
\[
\begin{array}{l}
AM=	\frac{1}{5}(120)+\frac{1}{5}(80)+\frac{1}{5}(90)+\frac{1}{5}(110)+
\frac{1}{5}(90) \\ 
= \frac{1}{5}(120+80+90+110+95) \\ 
= \frac{1}{5}(495) \\ 
= 99mmHg
\end{array}
\]

\end{frame}



% Slide-n
% Slide-n
\begin{frame}[t]{Population vs Sample Mean}
	\begin{center}
		\begin{tabular}{|c|c|} 
			\hline 
			Population & Sample \\ 
			\hline 
			$\mu  = \frac{\sum_{i=1}^{N} x_i }{N}$ & $\overline{x}  = 
			\frac{\sum_{i=1}^{n} 
			x_i }{n}$ \\
		 \hline
		$\mu$ = number of items in the population  & $\overline{x}$ = number of 
		items in the sample \\ 
		\hline 
		\end{tabular}
	\end{center}
\end{frame}


% Slide-3.2.3
\begin{frame}[t]{Impact of Outliers}
	\textbf{Example:} Five systolic blood pressures (mmHg) (n = 6) \\ 
	120, 80, 90, 110, 95, 500
	
	$$
	Mean = 
	\frac{120+80+90+110+95+500}{6}
	= \frac{995}{6}
	= \fbox{165.83mmHg}
	$$
	
	$$
	Mean = \overline{x}  = \frac{\sum x_i }{n}
	$$
	
	\begin{itemize}
		\item $	\overline{x} = $ mean of a dataset
		\item $	x_i =$ data points 
		\item $	n =$ number of sample 
	\end{itemize}
\end{frame}



% Slide-3.2.4
\begin{frame}[t]{Median}
	\begin{itemize}
		\item Value such that 50% of data on
		either side
		\item Sort data, then use middle element
		\item For even number of data points,
		average two middle elements
		\item More robust to outliers than mean
		\item However does not consider every
		data point
		\item Makes sense for ordinal data (data
		that can be sorted)
	\end{itemize}
\end{frame}

% Slide-3.2.5
\begin{frame}[t]{Median of a Dataset: Odd Sample Size}
	\textbf{Example:} Find the median systolic blood pressures (mmHg) (n=5) \\ 
	120, 80, 90, 110, 95 \pause 
	\begin{enumerate}
		\item \textbf{Sort Data:}  80, 90, 95, 110, 120  \pause 
		\item \textbf{Find the Middle Value:}  \fbox{95}
	\end{enumerate}
\end{frame}


% Slide-3.2.5
\begin{frame}[t]{Median of a Dataset: Even Sample Size}
	\textbf{Example:} Find the median systolic blood pressures (mmHg) (n=6) \\ 
	120, 80, 90, 110, 95, 85 \pause 
	\begin{enumerate}
		\item \textbf{Sort Data:}  80, 85, 90, 95, 110, 120  \pause 
		\item \textbf{Compute the Average of Middle 2 Values:}  
		$\frac{90+95}{2} = 137.5 $ \pause 
		\item \textbf{Computed Mean is the Median:}  \fbox{137.5}
	\end{enumerate}
\end{frame}


% Slide-3.2.6
\begin{frame}[t]{Impact of Outliers}
	\textbf{Example:} Five systolic blood pressures (mmHg) (n = 5) \\ 
	120, 80, 90, 110, 500 \pause 
		\begin{enumerate}
		\item \textbf{Sort Data:}  80, 90,110, 120, 500  \pause 
		\item \textbf{Find the Middle Value:}  \fbox{110}
	\end{enumerate}
\end{frame}

% Slide-3.2.7
\begin{frame}[t]{Mode}
	\begin{itemize}
		\item Most frequent value in dataset
		\item Highest bar in histogram
		\item Winner in elections
		\item Typically used with categorical data
		\item Unlike mean or median, mode need not
		be unique
		\item Not great for continuous data
		\item Continuous data needs to be discretized
		and binned first
	\end{itemize}
\end{frame}

% Slide-3.2.8 
\begin{frame}[t]{Mode of a Dataset}
\begin{itemize}
	\item \textbf{Candidate:} Abul, Akhi, Babul, Bithi, Dabul, Doli
	\item \textbf{Votes:} 60, 20, 10, 40, 50, 30
\end{itemize} 
Mode represents the most frequent value in the data, so the winner is \fbox{60}

\end{frame}

% Slide-3.2.9 
\begin{frame}[t]{Other Measures of Central Tendency}
	\begin{itemize}
		\item Geometric mean
		\begin{itemize}
			\item[--] Great for summarizing ratios
			\item[--] Compound Annual Growth Rate
			(CAGR)
		\end{itemize}
	\end{itemize}

\begin{itemize}
	\item Harmonic mean
	\begin{itemize}
		\item[--] Great for summarizing rates
		\item[--] Resistors in parallel
		\item[--] P/E ratios in finance
	\end{itemize}
\end{itemize}
\end{frame}


% Slide-n
\begin{frame}[t]{Geometric Mean of a Dataset}
	\begin{itemize}
		\item The geometric mean is calculated as the $nth$ root of the 
		product 
		of all values, where $n$ is the number of values. 
		$$GM = 
		\sqrt{(x_1\times  x_2 \times … \times x_n)}$$
		
		\item For example, if the data contains only two values, the square 
		root of the product of the two values is the geometric mean. For three 
		values, the cube-root is used, and so on.
		\item When calculating the arithmetic mean, the values can be positive, 
		negative, or zero.
		
		\item The geometric mean is appropriate when the data contains values 
		with different units of measure, e.g. some measure are height, some are 
		dollars, some are miles, etc.
		\item The geometric mean does not accept negative or zero values, e.g. 
		all values must be positive.
	\end{itemize}
\end{frame}


\begin{frame}[t]{Harmonic Mean of a Dataset}
	\begin{itemize}
		\item The harmonic mean is calculated as the number of values $n$ 
		divided 
		by the sum of the reciprocal of the values (1 over each value).
		
		$$HM =  \frac{n}{(\frac{1}{x_1} + \frac{1}{x_2} + … + \frac{1}{x_n})}$$
		\item The harmonic mean is the appropriate mean if the data is 
		comprised of rates.
		
		\item Recall that a rate is the ratio between two quantities with 
		different measures, e.g. speed, acceleration, frequency, etc.
		\item The harmonic mean does not take rates with a negative or zero 
		value, e.g. all rates must be positive.
	\end{itemize}
\end{frame}


