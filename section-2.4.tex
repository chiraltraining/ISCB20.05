% Slide-n
\begin{frame}[t]{Measures of Spread}
	\begin{itemize}
		\item Range (max - min)
		\item Inter-quartile range (IQR)
		\item Standard deviation and variance
	\end{itemize}
\end{frame}

% Slide-n
\begin{frame}[t]{Minimum}
\textbf{Example:}	Five systolic blood pressures (mmHg) (n = 5) \\
	120, 80, 90, 110, 95
	\begin{itemize}
		\item Minimum Value = \fbox{80}
	\end{itemize}
\end{frame}

% Slide-n
\begin{frame}[t]{Maximum}
\textbf{Example:}	Five systolic blood pressures (mmHg) (n = 5) \\
	120, 80, 90, 110, 95
	\begin{itemize}
		\item Maximum Value = \fbox{120}
	\end{itemize}
\end{frame}

% Slide-n
\begin{frame}[t]{Range}
 \textbf{Example:}	Five systolic blood pressures (mmHg) (n = 5) \\
	120, 80, 90, 110, 95
	
	$$
		\fbox{Range = Maximum - Minimum}
	$$ 
	
	\begin{itemize}
		\item Maximum = 120
		\item Minimum = 80 
		\item 	$
		Range = 120 - 80 = \fbox{40}
		$ 
	\end{itemize}

\end{frame}


% Slide-n
\begin{frame}[t]{Impact of Outliers}
	\textbf{Example:}	Five systolic blood pressures (mmHg) (n = 6) \\
	120, 80, 90, 110, 95, 500
	
	$$
	\fbox{Range = Maximum - Minimum}
	$$ 
	
	\begin{itemize}
		\item Maximum = 500
		\item Minimum = 80 
		\item 	$
		Range = 500 - 80 = \fbox{420}
		$ 
	\end{itemize}
\end{frame}


% Slide-n
\begin{frame}[t]{Percentiles}
	
	\begin{itemize}
		\item Divides data into 100 equal parts
		\item The pth percentile P is the value that is greater than or equal
		to p percent of the observations.
		\item Common percentiles are
		\begin{itemize}
			\item[--] 25th
			\item[--] 50th
			\item[--] 75th
		\end{itemize}
	\end{itemize}
\end{frame}


% Slide-n
\begin{frame}[t]{Method for Calculating Percentiles}
	\begin{itemize}
		\item $P_{50}$ = $Q_2$ = middle observation
		\item $P_{25}$ = $Q_1$ = middle observation of the lower half of
		observations
		\item $P_{75}$ = $Q_3$ = middle observation of the upper half of
		observations
	\end{itemize}
\end{frame}


% Slide-n
\begin{frame}[t]{Method for Calculating Percentiles}
	\textbf{Odd Observations}
	\begin{itemize}
		\item $P_{50}$ = $Q_2$ = middle observation
		\item $P_{25}$ = $Q_1$ = middle observation of the lower half of
		observations
		\item $P_{75}$ = $Q_3$ = middle observation of the upper half of
		observations
	\end{itemize}


	\textbf{Even Observations}
\begin{itemize}
	\item $P_{50}$ = $Q_2$ = average of the middle two observations
	\item $P_{25}$ = $Q_1$ = middle observation of the lower half of n/2
	observations
	\item $P_{75}$ = $Q_3$ = middle observation of the upper half of n/2
	observations
\end{itemize}
\end{frame}

% Slide-n
\begin{frame}[t]{Percentiles: Examples--1}
\textbf{Problem-1:} Sample height(cm) of 9 graduate students 168, 170, 150, 
160, 182, 140, 175, 180, 170(odd observations)
\end{frame}


\begin{frame}[t]{Percentiles: Examples--2}
	\textbf{Problem-2:} Sample height(cm) of 10 graduate students 168, 170, 
	150, 
	160, 182, 140, 175, 180, 170, 190(even observations)
\end{frame}



% Slide-n
\begin{frame}[t]{Inter Quartile Range(IQR)}
	$$
	IQR = Q_3 - Q_1
	$$
\end{frame}

% Slide-n
\begin{frame}[t]{Why IQR?}
	The primary advantage of using the interquartile range rather than the 
	range for the
	measurement of the spread of a data set is that the interquartile range is 
	not sensitive to outliers.
	
	\textbf{Example:}	Five systolic blood pressures (mmHg) (n = 6) \\
	120, 80, 90, 110, 95, 500
\end{frame}

% Slide-n
\begin{frame}[t]{Outlier Detection}
	\textbf{Example:}	Five systolic blood pressures (mmHg) (n = 6) \\
	120, 80, 90, 110, 95, 500
	
	$$
	[Q_1 - 1.5IQR, Q3+1.5IQR]
	$$
	
	
	
\end{frame}


\begin{frame}[t]{Five Number Summary}
	\textbf{Dataset:} Sample height(cm) of 10 graduate students 168, 170, 
	150, 160, 182, 140, 175, 180, 170, 190
	
	\begin{itemize}
		\item Min
		\item $Q_1$
		\item $Q_2$ or Median or 50th Percentile
		\item $Q_3$
		\item Max
	\end{itemize}
\end{frame}




% Slide-n
\begin{frame}[t]{Variance}
	\textbf{Dataset:} Sample height(cm) of 10 graduate students 168, 170, 
	150, 160, 182, 140, 175, 180, 170, 190
	\begin{enumerate}
		\item Calculate the center value/mean 
		\item Subtract each value from the mean and square all of them
		\item Calculate the sum of squared values 
		\item Divide the sum by the number of values 
	\end{enumerate}
\end{frame}

\begin{frame}[t]{Population vs Sample Variance}
	\begin{center}
		\begin{tabular}{|c|c|} 
			\hline 
			Population & Sample \\ 
			\hline 
			$\sigma^2  = \frac{\sum_{i=1}^{n} (x_i - \overline{x}) }{n}$ & 
			$ s^2  = 
			\frac{\sum_{i=1}^{n} 
				(x_i - \overline{x}) }{n-1}$ \\
			\hline
			$\sigma^2$ =  population variance  & $s^2$ = sample variance\\ 
			\hline 
		\end{tabular}
	\end{center}
\end{frame}


% Slide-n
\begin{frame}[t]{Standard Deviation}
	\textbf{Dataset:} Sample height(cm) of 10 graduate students 168, 170, 
	150, 160, 182, 140, 175, 180, 170, 190
	$$
	SD = \sqrt{Variance}
	$$ 
\end{frame}

% Slide-n
\begin{frame}[t]{Summary Statistics}
	\textbf{Dataset:} Sample height(cm) of 10 graduate students 168, 170, 
	150, 160, 182, 140, 175, 180, 170, 190
	
	\begin{itemize}
		\item Min
		\item $Q_1$ or 25th Percentile 
		\item $Q_2$ or Median or 50th Percentile
		\item $Q_3$ or 75th Percentile 
		\item Max
		\item Mean 
		\item Standard Deviation
	\end{itemize}
\end{frame}



